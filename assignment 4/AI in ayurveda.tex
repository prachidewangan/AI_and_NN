\documentclass{article}
	\usepackage[utf8]{inputenc}
	
	\title{\textbf{ Artificial Intelligence in Ayurveda}}
	\author{Prachi Dewangan \\ National Institute of Technology, Raipur}
	\date{August 2021}
	
	\begin{document}
	\maketitle
	\section*{Summary}
	Ayurveda is a system of alternative medicine that has its origins in the Indian subcontinent. The Indian Medical Association brands Ayurvedic practitioners who profess to practice medicine as quacks. Ayurveda is widely practiced in India and Nepal, with about 80 of the population reporting using it. Over the course of two millennia, Ayurvedic remedies have changed and evolved. Medicines, specific diets, meditation, yoga, massage, laxatives, enemas, and medical oils are examples of therapies. The majority of medicines are made up of complex plant components, minerals, and metals (perhaps under the influence of early Indian alchemy or rasa shastra). Ancient Ayurveda texts also taught surgical techniques, including rhinoplasty, kidney stone extractions, sutures, and the extraction of foreign objects.\\
	
	
	Artificial Intelligence is no longer a hidden domain, since it now provides enterprises with a wide range of capabilities. We all know that technology offers enormous promise for reducing human workloads, increasing storage capacity, lowering expenses, and much more. However, its use in Ayurveda may be indigestible for some persons. That’s because, Ayurveda regarded as something primitive and obsolete in compare to AI, which came into the view a few years back and is regarded as revolutionary and futuristic. \\
By drastically improving user experiences, AI may potentially be utilized to grow the Ayurveda market. Additionally, it can be used to support economic applications that have a significant impact on cost reduction, revenue growth, and asset utilization. \\
	\end{document}
