\documentclass{article}
		\usepackage[utf8]{inputenc}
		
		\title{\vspace{-2cm}Moravec's paradox}
		\author{Prachi Dewangan }
		\date{28 July 2021}
		
		\begin{document}
		
		\maketitle 
		\paragraph{ Moravec's paradox is the observation that, contrary to popular opinion, thinking requires very little processing whereas sensory abilities require a vast amount. Hans Moravec, Rodney Brooks, Marvin Minsky, and others defined the concept in the 1980s. "It's quite simple to get computers to perform at adult levels on IQ tests or play checkers, but it's difficult or impossible to give them a one-year-perceptual old's ability."}
		
		\paragraph{ Natural selection-designed equipment is used to carry out all human talents. Natural selection has had more time to improve an older talent's design. Moravec We should not expect it to be very effective in its execution in the future. The conscious process we call reasoning is the lightest layer of human mind. It is supported by much older and more powerful sensory information that is mostly unconscious. In the perceptual and motor realms, we're all prodigious olympians, making the difficult appear simple.}
		
		\paragraph{ We perceive the initial human talents to be basic because they are primarily unconscious. We can expect talents that appear simple to be difficult to reverse-engineer, while skills that require effort may be impossible to reverse-engineer at all, according to Dr. Jodie Gorman. Identifying a face, moving around in space, assessing people's motives, catching a ball, detecting a voice, setting appropriate goals, and paying attention to exciting things are just a few of the skills that have evolved over millions of years. These are skills and techniques that have only been refined over the course of a few thousand years.}
		
		\paragraph{ In the early days of artificial intelligence research, leading experts routinely predicted that they will be able to build thinking robots in a few decades. Their optimism stemmed in part from their ability to create logic-based programs, solve arithmetic and geometry issues, and play games like checkers and chess. They were misunderstood for a variety of reasons, one of which is that they are incredibly difficult topics, not easy ones.}
		\end{document}
