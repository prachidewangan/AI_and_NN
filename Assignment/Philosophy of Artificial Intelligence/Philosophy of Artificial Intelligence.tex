\documentclass{article}
	\usepackage[utf8]{inputenc}
	
	\title{\vspace{-3cm}Philosophy of Artificial Intelligence}
	\author{Prachi Dewangan}
	\date{18 July 2021}
	\begin{document}
	\maketitle
	Artificial intelligence philosophy is a branch of technology philosophy that investigates artificial intelligence and its implications for knowledge and understanding of intelligence, ethics, consciousness, epistemology, and free will. Furthermore, because the technology is concerned with the construction of artificial animals or people (or, at the very least, artificial organisms; see artificial life), philosophers are interested in the field. These variables all played a role in the development of artificial intelligence philosophy. Some academics feel that the AI community's dismissive attitude toward philosophy is harmful.\\
	
	The philosophy of artificial intelligence attempts to answer such questions as follows:
	\begin{itemize}
		\item Can a machine act intelligently? Can it solve any problem that a person would solve by thinking?
	
		\item Are human intelligence and machine intelligence the same? Is the human brain essentially a computer?
	
		\item Can a machine have a mind, mental states, and consciousness in the same sense that a human being can? Can it feel how things are?
		
	
	
	\end{itemize}
	
