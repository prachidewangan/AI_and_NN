\documentclass{article}
	\usepackage[utf8]{inputenc}
	
	\title{\vspace{-3cm}Philosophy of Artificial Intelligence}
	\author{Prachi Dewangan}
	\date{18 July 2021}
	\begin{document}
	\maketitle
	Artificial intelligence philosophy is a branch of technology philosophy that investigates artificial intelligence and its implications for knowledge and understanding of intelligence, ethics, consciousness, epistemology, and free will. Furthermore, because the technology is concerned with the construction of artificial animals or people (or, at the very least, artificial organisms; see artificial life), philosophers are interested in the field. These variables all played a role in the development of artificial intelligence philosophy. Some academics feel that the AI community's dismissive attitude toward philosophy is harmful.\\
	
	The philosophy of artificial intelligence attempts to answer such questions as follows:
	\begin{itemize}
		\item Can a machine act intelligently? Can it solve any problem that a person would solve by thinking?
	
		\item Are human intelligence and machine intelligence the same? Is the human brain essentially a computer?
	
		\item Can a machine have a mind, mental states, and consciousness in the same sense that a human being can? Can it feel how things are?
		
	
	
	\end{itemize}
	
	Is it feasible to build a machine that can solve all of the issues that people solve with intelligence? This question determines the extent of what machines may be able to accomplish in the future, and neurology directs AI research. "Every component of learning or any other attribute of intelligence may be so clearly defined that a computer can be created to imitate it," said a Dartmouth workshop proposal in 1956, summarising the core stance of most AI researchers.
	To address the question, the first step is to define "intelligence." The difficulty of defining intelligence was simplified by Alan Turing to a simple conversational inquiry.He claims that if a machine can respond to every question posed to it in the same language as a human would, then we may label it intelligent. Turing's test applies this polite norm to machines: if a computer acts intelligently, it is intelligent.\\
	This is a philosophical topic that is linked to the problem of other minds as well as the difficult problem of consciousness. The debate centres on John Searle's "strong AI" stance, which says that "a physical symbol system can have a mind and mental states." This is distinct from what Searle referred to as "weak AI," in which a physical symbol system may behave intelligently. Searle used the words to distinguish between strong and weak AI, allowing him to focus on what he considered to be the more fascinating and disputed topic. Even if we assumed we had a computer programme that behaved precisely like a human mind, he said, there would still be a tough philosophical matter to resolve.\\
	
	Artificial Intelligence encompasses a wide range of topics:
	
	\begin{enumerate}
		\item \textbf{Intelligent Agent: } An intelligent agent is a piece of software that can make decisions or deliver services based on its environment, human input, and previous experiences. These applications can be used to collect data on a regular, pre-programmed schedule or when prompted in real time by the user.
	
		\item \textbf{Problem Solving: }It is a subset of artificial intelligence that includes a variety of problem-solving strategies such as trees and heuristic algorithms. A problem-solving agent can alternatively be described as a goal-oriented agent who is always focused on achieving the desired outcomes.
	
		\item \textbf{Knowledge}: It is domain-specific information that can be used to solve problems within that area. We must decide how knowledge will be represented as part of building a software to address problems. The form of knowledge used by an agent is called a representation scheme.
	
		\item \textbf{Reasoning: }The reasoning is the mental process of deriving logical conclusion and making predictions from available knowledge, facts, and beliefs.
	
		\item \textbf{Planning:} It's about the decision-making duties that robots or computer programs conduct in order to attain a specific goal. The execution of planning entails selecting a set of actions that has a high probability of completing the job at hand.
	
		\item \textbf{Uncertain knowledge}: When the available knowledge has multiple causes leading to multiple effects or incomplete knowledge of causality in the domain.
	
		\item \textbf{Machine Learning: }It is a process that improves the knowledge of an AI program by making observations about its environment.
	
		\item \textbf{Communicating:} 8.	Natural language processing comprises technology such as human language machine translation, spoken conversation systems such as Siri, algorithms capable of producing publishable journalistic material, and social robots, all of which are designed to engage with humans in a human-like manner.'
	
		\item \textbf{Perception:} In Artificial Intelligence it is the process of interpreting vision, sounds, smell, and touch. Perception is a process to interpret, acquire, select, and then organize the sensory information from the physical world to make actions like humans.
	
		\item \textbf{Acting:} It refers to the action done by the algorithm in real world based on the processed data the algorithm has been fed.\end{enumerate}
		
	\end{document}
	
	
	
	\end{document}
